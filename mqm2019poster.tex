% !TeX document-id = {e06c1009-e01a-4844-9da1-2c8789db7fce}
% !TeX program = lualatex
% !TeX TXS-program:bibliography = txs:///biber

\documentclass[final, xcolor={svgnames}]{beamer}

% ====================
% Packages
% ====================

% Font, theme, and Beamerposter
\usepackage[T1]{fontenc}

\usepackage{lmodern}

\usepackage[orientation=portrait, size=a0, scale=1.1]{beamerposter}

\usetheme{BCHCambridge}

\usecolortheme{BCHCambridge}

\newcommand\bmmax{2}

\usepackage{bm}


% Chemistry
\usepackage[version=4]{mhchem} % Further chemistry typesetting support


% Scientific typesetting
\usepackage{siunitx}
\AtBeginDocument{\sisetup{math-rm=\mathrm, text-rm=\rmfamily}}


% Micro-typography
\usepackage[
	activate={true,nocompatibility},%
	final,%
	tracking=true,%
	factor=1100,%
	stretch=10,%
	shrink=10%
]{microtype}
\SetTracking{encoding={*}, shape=sc}{40} % Reduce spacing between sc characters
\microtypecontext{spacing=nonfrench}


% Maths support
\usepackage{amsmath, amssymb, mathtools, mathrsfs, braket, amsthm} % packages for further maths support

\usepackage{mleftright} %loading package for ensuring correct spacing before brackets
\mleftright


% Bibliography
\usepackage[%
	sorting=none,%
	style=nature,%
	articletitle=false,%
	autocite=superscript,%
	dateabbrev=false,%
	url=false,%
	isbn=false,%
	backend=biber%
]{biblatex}
\addbibresource{bib/mqm2019poster.bib}


% Tables
\usepackage{booktabs, multirow, tabularx}

\usepackage{lscape}


% Cross-referencing
\usepackage[capitalise, noabbrev]{cleveref} %loading package for enhanced cross-referencing; cleverref must be loaded after hyperref


% Diagrams and captions
\usepackage{graphicx}

\usepackage{caption, subcaption}


% pgfplots
%% We need the external library to use the \tikzsetexternalprefix variable
%% in all cases. However, we only need version compatibility and the rest of
%% the libraries if we actually need to compile pikz diagrams.

\usepackage{pgfplots} % pgfplots loads tikz automatically

\usetikzlibrary{external}

\tikzexternalize

\newif\iftikzex
\tikzextrue
%\tikzexfalse

\iftikzex
	\pgfplotsset{compat=1.16}
	
	\usetikzlibrary{calc, luamath, positioning, pgfplots.groupplots}
	
	\pgfplotscreateplotcyclelist{coloronly}{%
		{red},%
		{Blue},%
		{black!60!green},%
		{black!20!orange},%
		{green!30!brown},%
		{Blue!40!red},%
		{black!60!Blue},%
		{black!40!yellow},%
		{red!50!pink},%
		{green!70!Blue},%
	}
\fi

% Tikzexternalize
\makeatletter
	\newcommand*{\useexternalfile}[1]{%
		\iftikzex
			\tikzsetnextfilename{tikzoutput/#1-output}%
			\scalebox{1}{\input{\tikzexternal@filenameprefix#1.tikz}}
		\else
			\includegraphics[scale=1]{\tikzexternal@filenameprefix tikzoutput/#1-output.pdf}
		\fi
	}
\makeatother

\tikzsetexternalprefix{./tikz/}

% ====================
% Lengths
% ====================

% If you have N columns, choose \sepwidth and \colwidth such that
% (N+1)*\sepwidth + N*\colwidth = \paperwidth
\newlength{\sepwidth}
\newlength{\colwidth}
\setlength{\sepwidth}{0.0275\paperwidth}
\setlength{\colwidth}{0.45875\paperwidth}

\newcommand{\separatorcolumn}{\begin{column}{\sepwidth}\end{column}}

% ====================
% Title
% ====================

\title{Exploiting Multiple Symmetry-Broken SCF Solutions\\ to Describe Ground and Excited States of\\ Transition-Metal Complexes}

\author{Bang C. Huynh\inst{1} \and Alex J. W. Thom\inst{1}}

\institute[Chemistry, Cambridge, UK]{\inst{1} Department of Chemistry, University of Cambridge, United Kingdom}

% ====================
% Body
% ====================

\begin{document}

\begin{frame}[t]
\begin{columns}[t]
	\separatorcolumn
	
	\begin{column}{\dimexpr(2\colwidth+\sepwidth)}
		\begin{alertblock}{Low-Lying UHF Solutions and NOCI Wavefunctions of Model Octahedral \ce{[VF6]^{3-}}}
			\begin{figure}
				\begin{subfigure}[t]{0.49\textwidth}
					\centering
					\useexternalfile{d2_MS1_allnoci}
				\end{subfigure}
				\hfill
				\begin{subfigure}[t]{0.49\textwidth}
					\centering
					\useexternalfile{d2_MS0_allnoci}
				\end{subfigure}
				\captionsetup{justification=centering}
				\caption{
					Energy and symmetry of low-lying UHF solutions and NOCI wavefunctions constructed from them in octahedral \ce{[VF6]^3-}.\\[6pt]
					\footnotesize \tikz[baseline]{\node[draw, Gray!60!Black, inner sep = 3pt, anchor = base] {$\mathrm{S}_{M_S}$};}: symmetry-conserved UHF set $\mathrm{S}$ with $\hat{S}_z$ eigenvalue $M_S$. \tikz[baseline]{\node[draw, Gray!60!Black, densely dotted, inner sep = 3pt, anchor = base] {$\mathrm{S}_{M_S}$};}: spatial or spin symmetry-broken UHF set $\mathrm{S}$ with $\hat{S}_z$ eigenvalue $M_S$.\\[6pt]
					\footnotesize  $\Gamma[\mathrm{A}\oplus\mathrm{B}\oplus\mathrm{C}]$: a specific NOCI set of symmetry $\Gamma$ constructed from all of $\mathrm{A}$, $\mathrm{B}$, and $\mathrm{C}$. $\Gamma[\mathrm{A}, \mathrm{B}, \mathrm{C}]$: multiple NOCI sets of symmetry $\Gamma$ constructed from all non-trivial combinations of $\mathrm{A}$, $\mathrm{B}$, and $\mathrm{C}$.
				}
				\label{fig:d2_allnoci}
			\end{figure}
		\end{alertblock}
	\end{column}

	\separatorcolumn
\end{columns}
	
\begin{columns}[t]
	\separatorcolumn

	\begin{column}{\colwidth}
	
		\begin{block}{Introduction}
			Transition-metal complexes are \textbf{\color{Blue} strongly correlated} as they have many low-energy electronic states that exhibit \textbf{\color{Blue} high degrees of degeneracy}. \Cref{fig:corrTSd2} gives such states for octahedral $d^2$ as an example.
			
				\begin{figure}
					\begin{subfigure}[b]{0.49\textwidth}
						\centering
						\useexternalfile{corrd2}
						\caption{Correlation diagram (not to scale)}
					\end{subfigure}
					\hfill
					\begin{subfigure}[b]{0.49\textwidth}
						\centering
						\useexternalfile{TSd2}
						\caption{Tanabe--Sugano diagram}
					\end{subfigure}
					\caption{All electronic terms of a true $d^2$ system in an octahedral field.}
					\label{fig:corrTSd2}
				\end{figure}
				
			The non-linear HF equations for these complexes therefore admit \textbf{\color{Blue} multiple low-lying solutions} that are \textbf{\color{Blue} degenerate} or \textbf{\color{Blue} nearly degenerate}.
			
			We have located these solutions using \textbf{\color{Blue} SCF metadynamics} and investigated their \textbf{\color{red} symmetry properties} in a model octahedral $d^2$ system, \ce{[VF6]^{3-}} (\cref{fig:d2_allnoci}).
	  	\end{block}
	
		\begin{alertblock}{Symmetry Breaking in HF}
			Degenerate eigenfunctions of the spinless Hamiltonian \emph{must} transform as a single irreducible representation (irrep) of the underlying point group $\mathcal{B}$, the spin rotation group $\mathsf{SU}(2)$, or the time reversal group $\mathcal{T}$.

			HF wavefunctions do not have to obey this due to their approximate nature. Thus, consider a set of degenerate HF solutions %
				$\mathrm{S} =
					\left\lbrace
						\prescript{w}{}{\Psi} %
						\ | \ %
						w = 1, 2, \ldots %
					\right\rbrace%
				$ %
			and a particular group $\mathcal{G}$:
				\begin{itemize}
					\item if $\mathrm{S}$ spans a \textbf{\color{Blue} single irrep} in $\mathcal{G}$, then $\mathrm{S}$ is \textbf{\color{red} symmetry-conserved} in $\mathcal{G}$;
					\item if $\mathrm{S}$ spans \textbf{\color{Blue} multiple
					irreps} in $\mathcal{G}$, then $\mathrm{S}$ is \textbf{\color{red} symmetry-broken} in $\mathcal{G}$.
				\end{itemize}

			HF solutions break symmetry to become lower in energy and possibly recover some electron correlation. \textbf{\color{Blue} Restoring symmetry} of symmetry-broken HF solutions allows us to form \textbf{\color{red} physically meaningful wavefunctions} while \textbf{\color{red} incorporating said correlation}.
		\end{alertblock}
	
		\begin{block}{Restoring Symmetry: Non-Orthogonal CI (NOCI)}
			For a complete symmetry-broken set %
				$\mathrm{S} =
					\left\lbrace
						\prescript{w}{}{\Psi} %
						\ | \ %
						w = 1, 2, \ldots %
					\right\rbrace%
				$, %
			solving the generalised eigenvalue equation
				\begin{equation*}
					\boldsymbol{HA} = \boldsymbol{SAE} \quad %
					\textsf{where} \quad %
					\left(\boldsymbol{H}\right)_{wx} = %
						\braket{\prescript{w}{}{\Psi}|\hat{\mathscr{H}}|\prescript{x}{}{\Psi}} \  %
					\textsf{and} \  %
					\left(\boldsymbol{S}\right)_{wx} = %
						\braket{\prescript{w}{}{\Psi}|\prescript{x}{}{\Psi}} %
				\end{equation*}
			gives coefficients $A_{wm}$ such that the \textbf{\color{Blue} NOCI wavefunctions}
				\begin{equation*}
					\prescript{m}{}{\Phi} = \sum_{w} \prescript{w}{}{\Psi} A_{wm}
				\end{equation*}
			\textbf{\color{red} conserve symmetry} and can be used to approximate corresponding electronic terms.
		\end{block}
	
	\end{column}
	
	\separatorcolumn
	
	\begin{column}{\colwidth}
	
		\begin{block}{UHF vs. NOCI: Jahn--Teller Distortion}
			Consider the $T_{1g} \otimes e_g$ Jahn--Teller distortion in octaheral \ce{[VF6]^{3-}}. \Cref{fig:JT} shows that:
				\begin{itemize}
					\item the UHF $\mathrm{A}_1$ and $\mathrm{A}'_1$ solutions fail to exhibit the expected energy minima;
					\item the $\prescript{3}{}{T}_{1g}[\mathrm{A}_1 \oplus \mathrm{A}'_1]$ \textbf{\color{Blue} NOCI wavefunctions} do \textbf{\color{Red} show the expected stabilisation} and give the \textbf{\color{Red} correct degeneracy splitting} upon symmetry descent.
				\end{itemize}
			
			
			
			\begin{figure}
				\centering
				\useexternalfile{A.JT.Eg.minimal}
				\captionsetup{justification=centering}
				\caption{
					State energy in the $T_{1g} \otimes e_g$ Jahn--Teller distortion of octahedral \ce{[VF6]^{3-}}.\\[6pt]
					\footnotesize Dashed curves: symmetry-broken UHF $\mathrm{A}_1$ or $\mathrm{A}'_1$ solutions. Solid curves: $\prescript{3}{}{T}_{1g}[\mathrm{A}_1 \oplus \mathrm{A}'_1]$ NOCI wavefunctions.}
				\label{fig:JT}
			\end{figure}
		\end{block}
		
		\begin{block}{Solution Topology: Euclidean Realisation of State Distances}
			% Interpreting state distances as Euclidean distances, 
			By realising the distance matrices between symmetry-broken HF solutions to give \textbf{\color{Blue} polytopes} showing their \textbf{\color{red} arrangements in Euclidean space} (\cref{fig:AAdashpolytopes}), we hope to gain insight into the nature of their symmetry breaking.
			
			\begin{figure}
				\begin{subfigure}[b]{0.3\textwidth}
					\centering
					\useexternalfile{AAdashpolytope.Egu.elongation.g8}
					\caption{$S_2 = \SI{-0.04}{\angstrom}$}
				\end{subfigure}
				\hfill
				\begin{subfigure}[b]{0.3\textwidth}
					\centering
					\useexternalfile{AAdashpolytope.Egu.compression.g0}
					\caption{$S_2 = 0$}
				\end{subfigure}
				\hfill
				\begin{subfigure}[b]{0.3\textwidth}
					\centering
					\useexternalfile{AAdashpolytope.Egu.compression.g8}
					\caption{$S_2 = \SI{0.04}{\angstrom}$}
				\end{subfigure}
				\caption{Three-dimensional projections of polytopes of $\mathrm{A}_1$ and $\mathrm{A}'_1$ solutions along the $e_{g}u$ distortion.}
				\label{fig:AAdashpolytopes}
			\end{figure}
		\end{block}
	
		\begin{alertblock}{In a Nutshell}
			\begin{itemize}
				\item There exist \textbf{\color{DarkGreen} many low-lying HF solutions} in transition-metal complexes.
				\item Some solutions \textbf{\color{DarkGreen} break symmetry}, spin or spatial or both.
				\item NOCI can \textbf{\color{DarkGreen} restore symmetry} to give \textbf{\color{DarkGreen} proper physical behaviours}, \textit{e.g.}, vibronic coupling.
				\item There are \textbf{\color{DarkGreen} clear patterns} in symmetry breaking that need further understanding.
			\end{itemize}
		\end{alertblock}
	
		\begin{block}{References}
			
			\AtNextBibliography{\footnotesize}
			\nocite{*}
			\printbibliography[title=none]
			
		\end{block}
	
	\end{column}
	
	\separatorcolumn
\end{columns}
\end{frame}

\end{document}
