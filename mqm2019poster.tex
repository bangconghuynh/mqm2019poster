% !TeX document-id = {3ec03efe-bd26-4a88-995a-24dc71ddaacf}
% !TeX program = lualatex

% Gemini theme
% https://github.com/anishathalye/gemini

\documentclass[final, xcolor={svgnames}]{beamer}

% ====================
% Packages
% ====================

% Font, theme, and Beamerposter
\usepackage[T1]{fontenc}

\usepackage{lmodern}

\usepackage[orientation=portrait, size=a0, scale=1.3]{beamerposter}

\usetheme{BCHCambridge}

\usecolortheme{BCHCambridge}

\newcommand\bmmax{2}

\usepackage{bm}


% Chemistry
\usepackage[version=4]{mhchem} % Further chemistry typesetting support


% Scientific typesetting
\usepackage{siunitx}


% Maths support
\usepackage{amsmath, amssymb, mathtools, mathrsfs, braket, amsthm} % packages for further maths support

\usepackage{mleftright} %loading package for ensuring correct spacing before brackets
\mleftright


% Tables
\usepackage{booktabs, multirow, tabularx}

\usepackage{lscape}


% Diagrams and captions
\usepackage{graphicx}

\usepackage{caption, subcaption}


% pgfplots
%% We need the external library to use the \tikzsetexternalprefix variable
%% in all cases. However, we only need version compatibility and the rest of
%% the libraries if we actually need to compile pikz diagrams.

\usepackage{pgfplots} % pgfplots loads tikz automatically

\usetikzlibrary{external}

\tikzexternalize

\newif\iftikzex
\tikzextrue
%\tikzexfalse

\iftikzex
	\pgfplotsset{compat=1.16}
	
	\usetikzlibrary{calc, luamath, positioning, pgfplots.groupplots}
	
	\pgfplotscreateplotcyclelist{coloronly}{%
		{red},%
		{blue},%
		{black!60!green},%
		{black!20!orange},%
		{green!30!brown},%
		{blue!40!red},%
		{black!60!blue},%
		{black!40!yellow},%
		{red!50!pink},%
		{green!70!blue},%
	}
\fi

% Tikzexternalize
\makeatletter
	\newcommand*{\useexternalfile}[1]{%
		\iftikzex
			\tikzsetnextfilename{tikzoutput/#1-output}%
			\scalebox{1}{\input{\tikzexternal@filenameprefix#1.tikz}}
		\else
			\includegraphics[scale=1]{\tikzexternal@filenameprefix tikzoutput/#1-output.pdf}
		\fi
	}
\makeatother

\tikzsetexternalprefix{./tikz/}

% ====================
% Lengths
% ====================

% If you have N columns, choose \sepwidth and \colwidth such that
% (N+1)*\sepwidth + N*\colwidth = \paperwidth
\newlength{\sepwidth}
\newlength{\colwidth}
\setlength{\sepwidth}{0.03\paperwidth}
\setlength{\colwidth}{0.455\paperwidth}

\newcommand{\separatorcolumn}{\begin{column}{\sepwidth}\end{column}}

% ====================
% Title
% ====================

\title{Exploiting Multiple Symmetry-Broken SCF Solutions\newline to Describe Ground and Excited States of\newline Transition\textendash Metal Complexes}

\author{Bang C. Huynh\inst{1} \and Alex J. W. Thom\inst{1}}

\institute[shortinst]{\inst{1} University of Cambridge, UK}

% ====================
% Body
% ====================

\begin{document}

\begin{frame}[t]
\begin{columns}[t]
	\separatorcolumn
	
	\begin{column}{\dimexpr(2\colwidth+\sepwidth)}
		\begin{block}{Low-Lying UHF Solutions and NOCI Wavefunctions in Model Octahedral \ce{[VF6]^{3-}}}
			\begin{figure}
				\begin{subfigure}[t]{0.49\textwidth}
					\centering
					\useexternalfile{d2_MS1_allnoci}
				\end{subfigure}
				\hfill
				\begin{subfigure}[t]{0.49\textwidth}
					\centering
					\useexternalfile{d2_MS0_allnoci}
				\end{subfigure}
				\captionsetup{justification=centering}
				\caption{
					Energy and symmetry of low-lying UHF solutions and NOCI wavefunctions constructed from them in octahedral \ce{[VF6]^3-}.\\[6pt]
					\tikz[baseline]{\node[draw, Gray, inner sep = 3pt, anchor = base] {$\mathrm{S}_{M_S}$};}: symmetry-conserved solution $\mathrm{S}$ with $\hat{S}_z$ eigenvalue $M_S$. \tikz[baseline]{\node[draw, Gray, densely dotted, inner sep = 3pt, anchor = base] {$\mathrm{S}_{M_S}$};}: spatial or spin symmetry-broken solution $\mathrm{S}$ with $\hat{S}_z$ eigenvalue $M_S$.\\[6pt]
					$\Gamma[\mathrm{A}\oplus\mathrm{B}\oplus\mathrm{C}]$: a specific NOCI set of symmetry $\Gamma$ constructed from all of $\mathrm{A}$, $\mathrm{B}$, and $\mathrm{C}$. $\Gamma[\mathrm{A}, \mathrm{B}, \mathrm{C}]$: multiple NOCI sets of symmetry $\Gamma$ constructed from all non-trivial combinations of $\mathrm{A}$, $\mathrm{B}$, and $\mathrm{C}$.
				}
			\end{figure}
		\end{block}
	\end{column}

	\separatorcolumn
\end{columns}
	
\begin{columns}[t]
\separatorcolumn

\begin{column}{\colwidth}

  \begin{block}{A block title}

    Some block contents, followed by a diagram, followed by a dummy paragraph.

%    \begin{figure}
%      \centering
%      \begin{tikzpicture}[scale=6]
%        \draw[step=0.25cm,color=gray] (-1,-1) grid (1,1);
%        \draw (1,0) -- (0.2,0.2) -- (0,1) -- (-0.2,0.2) -- (-1,0)
%          -- (-0.2,-0.2) -- (0,-1) -- (0.2,-0.2) -- cycle;
%      \end{tikzpicture}
%      \caption{A figure caption.}
%    \end{figure}

    Lorem ipsum dolor sit amet, consectetur adipiscing elit. Morbi ultricies
    eget libero ac ullamcorper. Integer et euismod ante. Aenean vestibulum
    lobortis augue, ut lobortis turpis rhoncus sed. Proin feugiat nibh a
    lacinia dignissim. Proin scelerisque, risus eget tempor fermentum, ex
    turpis condimentum urna, quis malesuada sapien arcu eu purus.

  \end{block}

  \begin{block}{A block containing a list}

    Nam vulputate nunc felis, non condimentum lacus porta ultrices. Nullam sed
    sagittis metus. Etiam consectetur gravida urna quis suscipit.

    \begin{itemize}
      \item \textbf{Mauris tempor} risus nulla, sed ornare
      \item \textbf{Libero tincidunt} a duis congue vitae
      \item \textbf{Dui ac pretium} morbi justo neque, ullamcorper
    \end{itemize}

    Eget augue porta, bibendum venenatis tortor.

  \end{block}

  \begin{alertblock}{A highlighted block}

    This block catches your eye, so \textbf{important stuff} should probably go
    here.

    Curabitur eu libero vehicula, cursus est fringilla, luctus est. Morbi
    consectetur mauris quam, at finibus elit auctor ac. Aliquam erat volutpat.
    Aenean at nisl ut ex ullamcorper eleifend et eu augue. Aenean quis velit
    tristique odio convallis ultrices a ac odio.

    \begin{itemize}
      \item \textbf{Fusce dapibus tellus} vel tellus semper finibus. In
        consequat, nibh sed mattis luctus, augue diam fermentum lectus.
      \item \textbf{In euismod erat metus} non ex. Vestibulum luctus augue in
        mi condimentum, at sollicitudin lorem viverra.
      \item \textbf{Suspendisse vulputate} mauris vel placerat consectetur.
        Mauris semper, purus ac hendrerit molestie, elit mi dignissim odio, in
        suscipit felis sapien vel ex.
    \end{itemize}

    Aenean tincidunt risus eros, at gravida lorem sagittis vel. Vestibulum ante
    ipsum primis in faucibus orci luctus et ultrices posuere cubilia Curae.

  \end{alertblock}

\end{column}

\separatorcolumn

\begin{column}{\colwidth}

  \begin{block}{A block containing some math}

    Nullam non est elit. In eu ornare justo. Maecenas porttitor sodales lacus,
    ut cursus augue sodales ac.

    $$
    \int_{-\infty}^{\infty} e^{-x^2}\,dx = \sqrt{\pi}
    $$

    Interdum et malesuada fames $\{1, 4, 9, \ldots\}$ ac ante ipsum primis in
    faucibus. Cras eleifend dolor eu nulla suscipit suscipit. Sed lobortis non
    felis id vulputate.

    \heading{A heading inside a block}

    Praesent consectetur mi $x^2 + y^2$ metus, nec vestibulum justo viverra
    nec. Proin eget nulla pretium, egestas magna aliquam, mollis neque. Vivamus
    dictum $\mathbf{u}^\intercal\mathbf{v}$ sagittis odio, vel porta erat
    congue sed. Maecenas ut dolor quis arcu auctor porttitor.

    \heading{Another heading inside a block}

    Sed augue erat, scelerisque a purus ultricies, placerat porttitor neque.
    Donec $P(y \mid x)$ fermentum consectetur $\nabla_x P(y \mid x)$ sapien
    sagittis egestas. Duis eget leo euismod nunc viverra imperdiet nec id
    justo.

  \end{block}

  \begin{block}{Nullam vel erat at velit convallis laoreet}

    Class aptent taciti sociosqu ad litora torquent per conubia nostra, per
    inceptos himenaeos. Phasellus libero enim, gravida sed erat sit amet,
    scelerisque congue diam. Fusce dapibus dui ut augue pulvinar iaculis.

    \begin{table}
      \centering
      \begin{tabular}{l r r c}
        \toprule
        \textbf{First column} & \textbf{Second column} & \textbf{Third column} & \textbf{Fourth} \\
        \midrule
        Foo & 13.37 & 384,394 & $\alpha$ \\
        Bar & 2.17 & 1,392 & $\beta$ \\
        Baz & 3.14 & 83,742 & $\delta$ \\
        Qux & 7.59 & 974 & $\gamma$ \\
        \bottomrule
      \end{tabular}
      \caption{A table caption.}
    \end{table}

    Donec quis posuere ligula. Nunc feugiat elit a mi malesuada consequat. Sed
    imperdiet augue ac nibh aliquet tristique. Aenean eu tortor vulputate,
    eleifend lorem in, dictum urna. Proin auctor ante in augue tincidunt
    tempor. Proin pellentesque vulputate odio, ac gravida nulla posuere
    efficitur. Aenean at velit vel dolor blandit molestie. Mauris laoreet
    commodo quam, non luctus nibh ullamcorper in. Class aptent taciti sociosqu
    ad litora torquent per conubia nostra, per inceptos himenaeos.

    Nulla varius finibus volutpat. Mauris molestie lorem tincidunt, iaculis
    libero at, gravida ante. Phasellus at felis eu neque suscipit suscipit.
    Integer ullamcorper, dui nec pretium ornare, urna dolor consequat libero,
    in feugiat elit lorem euismod lacus. Pellentesque sit amet dolor mollis,
    auctor urna non, tempus sem.

  \end{block}

  \begin{block}{References}

    \nocite{*}
    \footnotesize{\bibliographystyle{plain}\bibliography{bib/mqm2019poster}}

  \end{block}

\end{column}

\separatorcolumn
\end{columns}
\end{frame}

\end{document}
